\begin{table}[t]
  \centering
  \begin{tabular}{lccc}
    \hline\hline
     & TR$_0$ & TR$_1$ & TR$_2$ \\
    \hline
    input-file & sample-hmc0.input & sample-hmc2.input &
    sample-hmc3.input \\
    $L^3\times T$ & $4^3\times 4$ & $4^3\times 4$ &
    $4^3\times 4$ \\

    $S_\mathrm{G}$ & Wilson & TlSym & Iwasaki \\
    $\beta $       & $6.0$  & $3.3$ & $1.95$ \\
    $\kappa$       & $0.177$ & $0.17$ & $0.163260$ \\
    $2\kappa\mu_q$ & $0.177$ & $0.01$ & $0.002740961$ \\
    $2\kappa \bar\mu$ & $-$  & $0.1105$ & $-$ \\
    $2\kappa \bar\epsilon$ & $-$ &$0.0935$ & $-$ \\
    $\langle P\rangle$ & $0.62457(7)$ & $0.53347(17)$ & $0.5951(2)$ \\
    $\langle R\rangle$ & $-$ & $0.30393(22)$ & $0.3637(3)$ \\
    \hline\hline
  \end{tabular}
  \caption{}
  \label{tab:testruns}
\end{table}

The source code ships with a number of sample input files. They are
located in the {\ttfamily sample-input} sub-directory. They are small
volume $V=4^4$ test runs designated to measure for instance the
average plaquette values.

Such a testrun can be performed for instance on a scalar machine by
typing 
\begin{quote}
  {\ttfamily ./hmc\_tm -f sample-hmc0.input}\ .
\end{quote}
Depending on the environment you are running in, you may need to adjust
the input parameters to match the maximal run-time and so on. The
expected average plaquette values are quoted in
table~\ref{tab:testruns} and also in the sample input files.

\subsection{Benchmark Executable}

Another useful test executable is a benchmark code. It can be build by
typing {\ttfamily make benchmark} and it will, when run, measure the
performance of the Dirac operator. It can be run in the serial or
parallel case. It reads its input from a file {\ttfamily
  benchmark.input} and the relevant input parameters are the
following:
\begin{verbatim}
L = 4
T = 4
NrXProcs = 2
NrYProcs = 2
NrZProcs = 2
UseEvenOdd = yes
UseSloppyPrecision = no
\end{verbatim}
In case of even/odd preconditioning the performance of the hopping
matrix is evaluated, in case of no even/odd the performance of the
Dirac operator. The important part of the output of the code is as
follows
\begin{verbatim}
[...]

 (1429 Mflops [64 bit arithmetic])

communication switched off
 (2592 Mflops [64 bit arithmetic])

The size of the package is 36864 Byte
The bandwidth is 662.91 + 662.91   MB/sec
\end{verbatim}
The bandwidth is not measured directly but computed from the
performance difference among with and without communication and the
package size. In case of a serial run the output is obviously
reduced. 


%%% Local Variables: 
%%% mode: latex
%%% TeX-master: "main"
%%% End: 
